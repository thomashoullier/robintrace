\section{Functional description}
\textcolor{red}{TODO: Remove the purely API descriptions from this section.}

This section defines the program's objects and their associated
operations. The style is minimal and close to the computations. For
the rationale sustaining the computation and complementary information,
see the justification section (\cref{sec:justification}).

\subsection{base}
Some base types are useful throughout the program. These are detailed in this
section.

\paragraph{Vec3}
\lstinline{Vec3} $\in \mathbb{R}^3$ are vectors in 3D space. They may
represent points or directions. Depending on the context, they can be
implicitely considered to have unit norm.

\paragraph{Mat3}
\lstinline{Mat3} $\in \mathbb{R}^{3 \times 3}$ are 3D matrices. They
are used to represent rotation matrices.

\subsection{ray}
\lstinline{ray} objects are the centerpiece of the simulation. They must be
lightweight objects.  \lstinline{ray} holds a position and a unit vector in the
direction and orientation of the propagation of light:

\begin{itemize}
\item \lstinline{Vec3 p}: A point.
\item \lstinline{Vec3 v}: A unit vector, oriented by light propagation.
\end{itemize}

The interpretation of the data contained in a \lstinline{ray} is dependent
on the context, as they are expressed in a given \gls{LCS}.

In addition to their geometric definition, rays also hold a status code.
This code signals whether raytracing operations were successful, and if
not, which error case was encountered. There is no guarantee on the value
of the ray point and vector when the status code signals an error.

\begin{itemize}
\item \lstinline{int code}: Status code.
\end{itemize}

The status codes are defined in \cref{tab:ray-status-codes}.

\begin{table} \caption{\label{tab:ray-status-codes} Ray status codes.}
\begin{tabular}{| c | l |} \hline
\textbf{Code} & \textbf{Meaning} \\ \hline
0 & Success, the ray is valid.\\ \hline
3 & refract: \gls{TIR} \\ \hline
4 & transfer: ray is parallel to the new local plane. \\ \hline
5 & standard intersection: No intersection. \\ \hline
\end{tabular} \end{table}

\subsection{rop}
\lstinline{rop} are ray operations.

\subsubsection{reflect}
\lstinline{reflect} is a ray operation which applies the law of specular
reflection \cite{wiki:specular-reflection}. The normal vector
$\overrightarrow{N}$ is an input to the operation. There are no error cases. The
operation is illustrated on \cref{fig:reflect}.

\begin{equation}
\begin{bmatrix} l_r \\ m_r \\ n_r \end{bmatrix} =
\begin{bmatrix} l \\ m \\ n \end{bmatrix} - 2 \cdot
\overrightarrow{N} \cdot \left( \overrightarrow{N} \cdot
\begin{bmatrix} l \\ m \\ n \end{bmatrix} \right)
\end{equation}

\begin{figure} \caption{\label{fig:reflect} reflect operation quantities.}
\includesvg[height=.2\textheight, width=.9\textwidth, keepaspectratio]
           {images/shape/abstract-reflect.svg}
\end{figure}

\subsubsection{refract}
\lstinline{refract} is a ray operation applying the Snell law of refraction
\cite{wiki:snell-refraction}. We use Xavier Bec's formula (\cite{Marrs:2021}
p.105, \cite{Bec:1997}) for efficiency.  The operation is illustrated on
\cref{fig:refract}.

\begin{figure} \caption{\label{fig:refract} refract operation quantities.}
\includesvg[height=.2\textheight, width=.9\textwidth, keepaspectratio]
           {images/shape/abstract-refract.svg}
\end{figure}

Let,

\begin{itemize}
\item $n_1$ the incident medium refraction index,
\item $n_2$ the output medium refraction index,
\item $n_r = \frac{n_1}{n_2}$,
\item $\overrightarrow{i}$ the unit incident ray direction,
\item $\overrightarrow{N}$ the unit surface normal vector,
\item $\overrightarrow{t}$ the unit refracted ray direction.
\end{itemize}

\begin{equation}
\begin{split}
&c_1 = - \overrightarrow{i} \cdot \overrightarrow{N} \\
&w = n_r \cdot c_1 \\
&c_{2m} = (w - n_r) \cdot (w + n_r)
\end{split} \end{equation}

At this stage, if $c_{2m} < -1$, then we set the \gls{TIR} ray
error code and the computation stops. Otherwise we proceed with
the computation of the refracted ray direction.

\begin{equation} \label{eq:bec-formula}
\overrightarrow{t} = n_r \cdot \overrightarrow{i} +
(w - \sqrt{1 + c_{2m}}) \cdot \overrightarrow{N}
\end{equation}

\subsection{transfer} \label{sec:transfer}
A transfer defines a new \gls{LCS} to propagate the ray to.  The transfer
operation a change of basis and an intersection with the newly defined
local plane.

\paragraph{Definition}
Let,
\begin{itemize}
\item LCS1 the starting \gls{LCS} defined by its origin and basis vectors
$(A_1, \hat{x_1}, \hat{y_1}, \hat{z_1})$,
\item LCS2 the \gls{LCS} defined by the transfer operation, 
$(A_2, \hat{x_2}, \hat{y_2}, \hat{z_2})$.
\end{itemize}

The transfer operation is characterized by
\begin{itemize}
\item \lstinline{Mat3} $B$: A rotation matrix between the basis vectors of LCS1
and LCS2. $B$ is orthogonal by definition, hence $B^{-1} = B^\top$
\cite{wiki:rotation-matrix}.
\item \lstinline{Vec3} $\overrightarrow{D}$: A translation vector between
LCS1 and LCS2.
\end{itemize}

The coordinates of the origin and basis vectors of LCS2 are expressed in
the original LCS1 coordinates by the following relations.

\begin{equation} \begin{cases}
A_2 = B \cdot \overrightarrow{D} \\
\hat{x_2} = B \cdot \hat{x_1} \\
\hat{y_2} = B \cdot \hat{y_1} \\
\hat{z_2} = B \cdot \hat{z_1}
\end{cases} \end{equation}

Which is to say, LCS2 is obtained from LCS1 by first applying the rotation $B$
and then translating the origin by $\overrightarrow{D}$, with
$\overrightarrow{D}$ expressed in the rotated coordinates. The change of basis
is illustrated on \cref{fig:transfer-definition}.

\begin{figure} \caption{\label{fig:transfer-definition}
Definition of the transfer between LCS1 and LCS2.}
\includesvg[height=.2\textheight, width=.9\textwidth, keepaspectratio]
           {images/shape/transfer-definition.svg}
\end{figure}

\paragraph{Operation}
The \lstinline{transfer} operation can be decomposed as two successive
operations on the ray:

\begin{itemize}
\item A change of basis from LCS1 to LCS2.
\item A ray intersection with the local plane of LCS2.
\end{itemize}

Let a ray expressed in LCS1 with starting point and direction
$(P_1, \overrightarrow{V_1})$. We first operate a change of basis from LCS1
to LCS2, which gives the coordinates $(P_2, \overrightarrow{V_2})$.

\begin{equation} \begin{cases}
P_2 = \begin{bmatrix} x_2 \\ y_2 \\ z_2 \end{bmatrix}
    = B^{-1} \cdot P_1 - \overrightarrow{D} \\
\overrightarrow{V_2} = \begin{bmatrix} l_2 \\ m_2 \\ n_2 \end{bmatrix}
  = B^{-1} \cdot \overrightarrow{V_1}
\end{cases} \end{equation}

We signal an error in the case $n_2 = 0$. This case corresponds to the
ray being parallel to the local plane of LCS2. We operate the intersection
of the ray with the LCS2 local plane next. The result ray is
$(P_3, \overrightarrow{V_3})$. The operation is illustrated by
\cref{fig:transfer-operation}.

\begin{equation} \begin{cases}
\overrightarrow{V_3} = \overrightarrow{V_2} \\
t = - \frac{z_2}{n_2} \\
P_3 = \begin{bmatrix} x_2 + t \cdot l_2 \\
                      y_2 + t \cdot m_2 \\
                      0 \end{bmatrix}
\end{cases} \end{equation}

\begin{figure} \caption{\label{fig:transfer-operation}
Illustration of the transfer operation on a ray.}
\includesvg[height=.2\textheight, width=.9\textwidth, keepaspectratio]
           {images/shape/transfer-operation.svg}
\end{figure}

\subsection{shape}
Shape is an abstract concept specifying two operations:

\begin{itemize}
\item \lstinline{intersect}: Intersect a ray with the shape.
\item \lstinline{normal}: Provide a vector normal to the shape at the current
      ray position.
\end{itemize}

\paragraph{intersect}
The intersection operation takes a \lstinline{ray} expressed in the current
surface coordinate system with point on the local plane. It computes the
intended intersection point between the ray and the shape.  This operation is
illustrated on \cref{fig:abstract-rayinter}.

\begin{figure} \caption{\label{fig:abstract-rayinter} Illustration of the
abtract intersect operation for a shape on a ray.}
\includesvg[height=.2\textheight, width=.9\textwidth, keepaspectratio]
           {images/shape/abstract-rayinter.svg}
\end{figure}

\paragraph{normal}
The normal operation provides a normal vector at the ray's current
position on the shape. The normal vector is expressed in the surface
LCS. The normal vector is a unit vector. The normal vector is oriented
with a $\hat{z}$ component of opposite sign to that of the ray's vector,
\textit{ie} the normal vector is in the opposite half-plane to the incident
ray. The normal operation is illustrated on \cref{fig:abstract-normal}.

\begin{figure} \caption{\label{fig:abstract-normal} Illustration of the
abstract normal operation for a shape and a ray.}
\includesvg[height=.2\textheight, width=.9\textwidth, keepaspectratio]
           {images/shape/abstract-normal.svg}
\end{figure}

\subsubsection{plane}
\paragraph{Definition}
A \lstinline{plane} is the local $z=0$ plane in the current \gls{LCS}.
It is specified implicitely.

\paragraph{intersect}
The input ray is already on the local plane. We do \emph{nothing} and cannot
fail.

\paragraph{normal}
The plane normal vector is trivial.
It is $\overrightarrow{N} = (0, 0, -\textrm{sign}(n))$.
There are no error cases.

\subsubsection{standard}
The so-called \lstinline{standard} shape describes surfaces which
belong to quadrics of revolution with axis $z$.

\paragraph{Definition}
The surface shape \lstinline{standard} is defined using:
\begin{itemize}
\item $c$: the radius of curvature reciprocal, $c = \frac{1}{R}$.
\item $k$: a scalar parameter which changes the type of the quadric.
\end{itemize}

The mathematical classification of the surface depends on the value of
$k$,

\begin{itemize}
\item $k < -1$: One of the sheets of a hyperboloid of revolution of two sheets.
\item $k = -1$: Circular paraboloid.
\item $k > -1$: Spheroid, with the case $k=0$ being a sphere.
\end{itemize}

Note that the case $c=0$ describes a plane.

An explicit altitude formula may be given for part of the surface (in the case
of spheroids, only the hemisphere containing the apex is described by this
formula). Let $r^2 = x^2 + y^2$.

\begin{equation} \label{eq:standard-z}
z = \frac{c \cdot r^2}{1 + \sqrt{1 - (k + 1) \cdot c^2 \cdot r^2}}
\end{equation}

The effective surface definition is given by the \lstinline{intersect}
operation.  Please note that in the case of spheroids, intersections
beyond the hemisphere closest to the apex are valid.

\paragraph{intersect}
The intersection point $I$ is found with the following operations sequence.

\begin{equation} \begin{cases}
f = c \cdot ({x_P}^2 + {y_P}^2) \\
g = n - c \cdot (l \cdot x_P + m \cdot y_P) \\
h = g^2 - c \cdot f \cdot (1 + k \cdot n^2)
\end{cases} \end{equation}

In the case $h \leq 0$, we signal a ray error of absence of intersection.
Else we continue,

\begin{equation}
t = \frac{f}{g + \textrm{sign}(n) \cdot \sqrt{h}}
\end{equation}

\begin{equation}
I = \begin{bmatrix} x_P \\ y_P \\ 0 \end{bmatrix} + t \cdot
    \begin{bmatrix} l \\ m \\ n \end{bmatrix}
\end{equation}

\paragraph{normal}
The normal vector at the current point is given by,

\begin{equation}
\overrightarrow{N} =
\begin{bmatrix}
c \cdot x \\ c \cdot y \\ c \cdot (k+1) \cdot z - 1
\end{bmatrix} \cdot
\frac{\textrm{sign}(n)}{
\sqrt{1 - 2 c \cdot k \cdot z + c^2 \cdot (k+1) \cdot k \cdot z^2}}
\end{equation}

