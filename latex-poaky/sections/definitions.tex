\section{Definitions and conventions}
We outline the conventions we use. Some details may differ from the choices
made in other tools.

\subsection{Raytracing sequence}
Poaky deals with the elementary operations in optical systems represented as a
sequence of objects operating on rays of light. The objects are of
two main types, optical surfaces and geometric propagators through some medium
in between surfaces (which we call \emph{transfer}). The sequence in which rays
of light interact with either surfaces or transfers is known a priori. The
simulation of the propagation of rays through such a sequence of objects is
known as \emph{sequential raytracing}. This discipline is linked with the more
well known raytracing for rendering, though they seem to have developed more or
less autonomously.

\subsection{Local coordinate system} \label{sec:LCS}
Each surface corresponds to an implicit \gls{LCS}.
This coordinate system may be described as containing (\cref{fig:LCS}):

\begin{itemize}
\item An apex $A$, which is the origin.
\item A local $z=0$ plane, which we often refer to as the \emph{local plane}.
\item A set of axes (implicit).
\end{itemize}

\begin{figure} \caption{\label{fig:LCS} LCS diagram.}
\includesvg[height=.2\textheight, width=.9\textwidth, keepaspectratio]
           {images/conventions/LCS.svg}
\end{figure}

We call the \gls{LCS} implicit because the actual meaning of the data
expressed in it depends on the interplay between surface defition, ray
transfer equations and ray operation conventions.

\subsection{Rays in local coordinate systems}

\begin{figure} \caption{\label{fig:ray-in-LCS} Ray definition
in \gls{LCS}.}
\includesvg[height=.2\textheight, width=.9\textwidth, keepaspectratio]
           {images/conventions/ray_in_LCS.svg}
\end{figure}

Given a \gls{LCS}, a ray may be defined by (\cref{fig:ray-in-LCS}):
\begin{itemize}
\item $P = \begin{bmatrix}x \\ y \\ z \end{bmatrix}$, a point.
\item $\overrightarrow{V} = \begin{bmatrix} l \\ m \\ n \end{bmatrix}$, a unit
vector oriented by the light propagation.
\end{itemize}

The components of $\overrightarrow{V}$ are often called in optics by the name
\emph{direction cosines} \cite{mathworld:direction-cosine}. Indeed, given the
basis vectors of our \gls{LCS}: $(\hat{x}, \hat{y}, \hat{z})$, the vector
$\overrightarrow{V}$ can be described by the angles $(\alpha, \beta, \gamma)$
between itself and each basis vector respectively. These angles may be defined
by \cref{eq:direction-cosines}.

\begin{equation} \label{eq:direction-cosines}
\begin{cases}
l &= \cos(\alpha) = \frac{\overrightarrow{V} \cdot \hat{x}}
                          {\abs{\overrightarrow{V}}} \\
m &= \cos(\beta) = \frac{\overrightarrow{V} \cdot \hat{y}}
                         {\abs{\overrightarrow{V}}} \\
n &= \cos(\gamma) = \frac{\overrightarrow{V} \cdot \hat{z}}
                          {\abs{\overrightarrow{V}}}
\end{cases}
\end{equation}

The points $(x_t, y_t, z_t)$ describing the ray trajectory through the use of a parameter $t$
are described by \cref{eq:ray-trajectory}.

\begin{equation} \label{eq:ray-trajectory}
\begin{bmatrix}
x_t \\ y_t \\ z_t
\end{bmatrix} =
\begin{bmatrix}
x + t \cdot l \\
y + t \cdot m \\
z + t \cdot n
\end{bmatrix}
\end{equation}
