\documentclass[letterpaper]{article}
\title{Poaky documentation}
\author{Thomas HOULLIER} 

\usepackage[colorlinks=true, allcolors=blue,
            hyperfootnotes=false,
            pdfauthor={Thomas HOULLIER},
            pdftitle={Poaky documentation},
	    pdfkeywords={Optical design, Raytracing, C++}]
            {hyperref} % Links for ref/cite.

%% Loading packages
\usepackage{amsmath} % For cases in equations.
\usepackage{amsfonts} % For maths sets.
\usepackage{physics} % For \abs{} and \norm{}.
\usepackage[inkscapelatex=false]{svg} %svg graphics
\usepackage{siunitx} % units formatting

\usepackage[backend=biber,style=numeric,citestyle=numeric-comp,maxcitenames=99,dateabbrev=false]{biblatex}
\addbibresource{biblio.bib}
\usepackage{setspace} % Bibliography spacings
\DeclareSourcemap{
  \maps[datatype=bibtex]{
    \map[overwrite]{
      \step[fieldsource=doi, final]
      \step[fieldset=url, null]
      \step[fieldset=eprint, null] }}}
\setcounter{biburllcpenalty}{7000} % break long url in bibliography
\setcounter{biburlucpenalty}{8000}
\renewcommand*{\bibfont}{\footnotesize} % bibliography font size
% Format of biblatex urldate in the bibliography.
\DeclareFieldFormat{urldate}{%
  Visited on \thefield{urlday}\addspace%
  \mkbibmonth{\thefield{urlmonth}}\addspace%
  \thefield{urlyear}\isdot}
\usepackage[ruled,vlined]{algorithm2e} % Algorithms.
\DontPrintSemicolon
\SetKwInOut{Input}{Input}\SetKwInOut{Output}{Output}
\usepackage{mathtools} % Ceiling function.
\usepackage{outlines} % Nest lists.
\usepackage{interval} % Writing intervals.
\usepackage[font={footnotesize,sf}]{caption} %Caption for figures in minipages.
\usepackage{floatrow}
% Figure captions always below. Figures always centered.
\floatsetup[figure]{capposition=bottom,objectset=centering}
\usepackage{wrapfig} %Wrapping figure with text.
\usepackage{stmaryrd} % Double brackets for integers interval.
\usepackage{doi} % Hyperlink DOI
\usepackage{etoolbox} %Ragged right bibliography.
\usepackage{color, colortbl} % Coloring rows in tables.
\usepackage{subcaption} % Subfigures.
\usepackage{pdfpages} % Include PDF pages.
\usepackage{epigraph} % Quotations at beginning of chapters.
\setlength\epigraphwidth{.8\textwidth}
\usepackage[acronym,nonumberlist,nogroupskip,nopostdot]{glossaries} % Glossary for acronyms.
\renewcommand*{\glstextformat}[1]{\textcolor{black}{#1}} % No color on links for abbrev.

\DeclarePairedDelimiter{\ceil}{\lceil}{\rceil} % Ceiling function.
\DeclarePairedDelimiter{\floor}{\lfloor}{\rfloor} % Floor function.

\DeclareMathOperator*{\argmin}{argmin}

\setcounter{tocdepth}{3} % Table of content depth
\setcounter{secnumdepth}{3} % Section numbering depth

% Non-breaking around footnotes.
\makeatletter
\let\Footnote\footnote
\def\pst@@killglue{\unskip\ifdim\lastskip>\z@\expandafter\pst@@killglue\fi}
\def\footnote{\pst@@killglue\Footnote}
\makeatother

% More space below equations
\appto\normalsize{\belowdisplayshortskip=\belowdisplayskip}

% Rewrite month codes in bibliography
\DeclareSourcemap{
  \maps[datatype=bibtex]{
    \map[overwrite]{
      \step[fieldsource=month, match=\regexp{\A(j|J)an(uary)?\Z}, replace=1]
      \step[fieldsource=month, match=\regexp{\A(f|F)eb(ruary)?\Z}, replace=2]
      \step[fieldsource=month, match=\regexp{\A(m|M)ar(ch)?\Z}, replace=3]
      \step[fieldsource=month, match=\regexp{\A(a|A)pr(il)?\Z}, replace=4]
      \step[fieldsource=month, match=\regexp{\A(m|M)ay\Z}, replace=5]
      \step[fieldsource=month, match=\regexp{\A(j|J)un(e)?\Z}, replace=6]
      \step[fieldsource=month, match=\regexp{\A(j|J)ul(y)?\Z}, replace=7]
      \step[fieldsource=month, match=\regexp{\A(a|A)ug(ust)?\Z}, replace=8]
      \step[fieldsource=month, match=\regexp{\A(s|S)ep(tember)?\Z}, replace=9]
      \step[fieldsource=month, match=\regexp{\A(o|O)ct(ober)?\Z}, replace=10]
      \step[fieldsource=month, match=\regexp{\A(n|N)ov(ember)?\Z}, replace=11]
      \step[fieldsource=month, match=\regexp{\A(d|D)ec(ember)?\Z}, replace=12]}}}

% Footnotes marker color
\renewcommand\thefootnote{\textcolor{blue}{\arabic{footnote}}}

\pdfsuppresswarningpagegroup=1 % Silence warnings about pagegroups for figures.
\pdfminorversion=6 % PDF version 1.6 since we include articles in 1.6.

% Allow an extra pass to fix overfull hboxes by allowing more whitespace.
\emergencystretch=1em

% Page numbering and copyright notice.
\usepackage{fancyhdr}
\usepackage{lastpage}

\fancypagestyle{FirstPage}{
\fancyhf{} % Clear footer.
\rfoot{\thepage \hspace{1pt} of \pageref*{LastPage}}
\renewcommand{\headrulewidth}{0pt} % Remove rule at top of page
\lfoot{\href{https://creativecommons.org/licenses/by/4.0/}
       {\includesvg[inkscapelatex=false,height=14pt]{images/ccby.svg}}}}

\fancypagestyle{plain}{
\fancyhf{} % Clear footer.
\rfoot{\thepage \hspace{1pt} of \pageref*{LastPage}}
\renewcommand{\headrulewidth}{0pt} % Remove rule at top of page
}

% Version history
\usepackage{vhistory}

% Keywords
\providecommand{\keywords}[1]{\textbf{Keywords --} #1}

% Glossary
\makeglossaries
\loadglsentries{glossary/glossary.tex}

\usepackage{fontawesome} %inline icons
\usepackage{xcolor}
\usepackage{listings} % Code listings
\definecolor{codeback}{rgb}{0.99,0.99,0.98}
\definecolor{codecomment}{HTML}{0588fc}
\definecolor{codekeyword}{HTML}{af5f00}
\definecolor{codestring}{HTML}{ffa07a}
\lstdefinestyle{mystyle}{
  backgroundcolor=\color{codeback},
  commentstyle=\color{codecomment},
  keywordstyle=\color{codekeyword},
  stringstyle=\color{codestring},
  basicstyle=\ttfamily\footnotesize,
  breakatwhitespace=false,         
  breaklines=true,                 
  captionpos=b,                    
  keepspaces=true,                 
  numbers=left,                    
  numbersep=5pt,                  
  showspaces=false,                
  showstringspaces=false,
  showtabs=false,                  
  tabsize=2}
\lstset{style=mystyle}

\usepackage[capitalise,nameinlink]{cleveref} % Include eg. "Fig." in front of figures.
\crefname{algorithm}{Alg.}{Algs.}
\crefname{table}{Tab.}{Tabs.}
\crefname{equation}{Eq}{Eqs.}
% Equation cross-references.
%\creflabelformat{equation}{#2#1#3}
\crefformat{equation}{(#2Eq.\thinspace#1#3)}

% No parentheses in equation labels.
%\newtagform{noparen}{}{}
%\usetagform{noparen}



\begin{document}
\frenchspacing
\date{v1.0 -- \today}
\maketitle
\thispagestyle{FirstPage}

\begin{abstract}
Poaky is a low-level software API layer for RobinTrace. Poaky is a
reference implementation for ray-wise and element-wise operations in
the forward simulation of sequential raytracing. The software is
written in C++.
\end{abstract}

\keywords{Optical design, Sequential raytracing, C++}

\begin{versionhistory}
\vhEntry{1.0}{\today}{TH}{creation}
\end{versionhistory}
\setcounter{table}{0} % Reset the table counter.

\tableofcontents
\pagestyle{plain}

\section{Introduction}
We document Poaky, which is a software component of RobinTrace. Poaky deals
with basic ray-wise and element-wise sequential raytracing operations. Its
goal is to provide a reference implementation and document clearly the
forward simulation pass of sequential raytracing. The simulation is carried
out within the context of geometrical optics.

We had similar endeavours in a previous work \cite{Houllier-thesis}.
The present software is a rewrite of this work after putting more thought and
research into the problem. Much of the present work is ultimately a rehash of
the work of Welford \cite{Welford:1986}, who was foundational in computerized
optical design raytracing.

In its present version, only a handful of operations are implemented. The
current development goal is to reach a minimal working example in RobinTrace.
Whether Poaky will serve as the base layer for RobinTrace in its current state
or merely as a reference is not yet decided.

\section{Definitions and conventions}
We must provide the conventions that were used. They are choices
on our part and may differ from one optical design tool to the next.

\subsection{Local coordinate system}
Each surface corresponds to an implicit \gls{LCS}.
This coordinate system may be described as containing (\cref{fig:LCS}):

\begin{itemize}
\item An apex $A$, which is the origin.
\item A local $z=0$ plane, which we often refer to as the \emph{local plane}.
      \textcolor{red}{TODO: Add "local plane" in a glossary, pointing here.}
\item A set of axes (implicit).
\end{itemize}

\begin{figure} \caption{\label{fig:LCS} LCS diagram.}
\includesvg[width=.7\textwidth]{images/conventions/LCS.svg}
\end{figure}

We call the \gls{LCS} implicit because the actual meaning of the data
expressed in it depends on the interplay between surface defition, ray
transfer equations and ray operation conventions.

\subsection{Rays in local coordinate systems}

\begin{figure} \caption{\label{fig:ray-in-LCS} Ray definition
in \gls{LCS}.}
\includesvg[width=.7\textwidth]{images/conventions/ray_in_LCS.svg}
\end{figure}

Given a \gls{LCS}, a ray may be defined by (\cref{fig:ray-in-LCS}):
\begin{itemize}
\item $P = \begin{bmatrix}x \\ y \\ z \end{bmatrix}$, a point.
\item $\overrightarrow{V} = \begin{bmatrix} l \\ m \\ n \end{bmatrix}$, a unit
vector oriented by the light propagation.
\end{itemize}

The components of $\overrightarrow{V}$ are often called in optics by the name
\emph{direction cosines} \cite{mathworld:direction-cosine}. Indeed, given the
basis vectors of our \gls{LCS}: $(\hat{x}, \hat{y}, \hat{z})$, the vector
$\overrightarrow{V}$ can be described by the angles $(\alpha, \beta, \gamma)$
between itself and each basis vector respectively. These angles may be defined
by \cref{eq:direction-cosines}.

\begin{equation} \label{eq:direction-cosines}
\begin{cases}
\cos(\alpha) &= \frac{\overrightarrow{V} \cdot \hat{x}}
                     {\abs{\overrightarrow{V}}} \\
\cos(\beta) &= \frac{\overrightarrow{V} \cdot \hat{y}}
                    {\abs{\overrightarrow{V}}} \\
\cos(\gamma) &= \frac{\overrightarrow{V} \cdot \hat{z}}
                     {\abs{\overrightarrow{V}}}
\end{cases}
\end{equation}

The points $(x_t, y_t, z_t)$ describing the ray trajectory through the use of a parameter $t$
are described by \cref{eq:ray-trajectory}.

\begin{equation} \label{eq:ray-trajectory}
\begin{bmatrix}
x_t \\ y_t \\ z_t
\end{bmatrix} =
\begin{bmatrix}
x + t \cdot l \\
y + t \cdot m \\
z + t \cdot n
\end{bmatrix}
\end{equation}

\subsection{Ray transfer conventions}
\textcolor{red}{TODO:
1. Rays always on the local plane when transfered to the next surface.
2. Rays can be wherever in the LCS at before being transfered.}


\section{Functional description}
This section defines the program's objects and their associated
operations. The style is minimal and close to the computations. For
the rationale sustaining the computation and complementary information,
see the justification section (\cref{sec:justification}).

\subsection{base}
Some base types are useful throughout the program. These are detailed in this
section.

\subsubsection{Point3}
\lstinline{Point3} are points in 3D space. They are described by $(x, y, z)$
coordinates.

\subsubsection{UVec3}
\lstinline{UVec3} are unit vectors in 3D space.

\subsection{ray}
\lstinline{ray} objects are the centerpiece of the simulation. They must be
lightweight objects.  \lstinline{ray} holds a position and a unit vector in the
direction and orientation of the propagation of light:

\begin{itemize}
\item \lstinline{Point3 p}: A point.
\item \lstinline{UVec3 v}: A vector, oriented by light propagation.
\end{itemize}

The interpretation of the data contained in a \lstinline{ray} is dependent
on context, as they are expressed in the current surface coordinate system.

In addition to their geometric definition, rays also hold a status code.
This code signals whether raytracing operations were successful, and if
not, which error case was encountered.

\begin{itemize}
\item \lstinline{int code}: Status code.
\end{itemize}

The status codes are defined in \cref{tab:ray-status-codes}.

\begin{table} \caption{\label{tab:ray-status-codes} Ray status codes.}
\begin{tabular}{| c | l |} \hline
\textbf{Code} & \textbf{Meaning} \\ \hline
0 & Success, the ray is valid.\\ \hline
1 & Sphere intersection: No intersection. \\ \hline
2 & Sphere intersection: Intersection beyond first hemisphere. \\ \hline
3 & refract: \gls{TIR} \\
\hline \end{tabular}
\end{table}

\subsection{rop}
\lstinline{rop} are ray operations.

\subsubsection{reflect}
\lstinline{reflect} is a ray operation which applies the law of specular
reflection \cite{wiki:specular-reflection}. The ray direction is modified in
place. The normal vector $\overrightarrow{N}$ is already computed. There are
no error cases. The operation is illustrated on \cref{fig:reflect}.

\begin{equation}
\begin{bmatrix} l_r \\ m_r \\ n_r \end{bmatrix} =
\begin{bmatrix} l \\ m \\ n \end{bmatrix} - 2 \cdot
\overrightarrow{N} \cdot \left( \overrightarrow{N} \cdot
\begin{bmatrix} l \\ m \\ n \end{bmatrix} \right)
\end{equation}

\begin{figure} \caption{\label{fig:reflect} reflect operation quantities.}
\includesvg[height=.2\textheight, width=.9\textwidth, keepaspectratio]
           {images/shape/abstract-reflect.svg}
\end{figure}

\subsubsection{refract}
\lstinline{refract} is a ray operation applying the Snell law of refraction
\cite{wiki:snell-refraction}. The ray direction is modified in place.  We use
Xavier Bec's formula (\cite{Marrs:2021} p.105, \cite{Bec:1997}) for efficiency.
The operation is illustrated on \cref{fig:refract}.

\begin{figure} \caption{\label{fig:refract} refract operation quantities.}
\includesvg[height=.2\textheight, width=.9\textwidth, keepaspectratio]
           {images/shape/abstract-refract.svg}
\end{figure}

Let,

\begin{itemize}
\item $n_1$ the incident medium refraction index,
\item $n_2$ the output medium refraction index,
\item $n_r = \frac{n_1}{n_2}$,
\item $\overrightarrow{i}$ the unit incident ray direction,
\item $\overrightarrow{N}$ the unit surface normal vector,
\item $\overrightarrow{t}$ the unit refracted ray direction.
\end{itemize}

\begin{equation}
\begin{split}
&c_1 = - \overrightarrow{i} \cdot \overrightarrow{N} \\
&w = n_r \cdot c_1 \\
&c_{2m} = (w - n_r) \cdot (w + n_r)
\end{split} \end{equation}

At this stage, if $c_{2m} < -1$, then we set the \gls{TIR} ray
error code and the computation stops. Otherwise we proceed with
the computation of the refracted ray direction.
with:

\begin{equation}
\overrightarrow{t} = n_r \cdot \overrightarrow{i} +
(w - \sqrt{1 + c_{2m}}) \cdot \overrightarrow{N}
\end{equation}

\textcolor{red}{TODO:
\begin{itemize}
\item transfer operation
\end{itemize}}

\subsection{shape}
Shapes are an abstract concept specifying two operations:

\begin{itemize}
\item \lstinline{intersect}: Intersect a ray with the shape.
\item \lstinline{normal}: Provide a normal vector at the current ray
      position.
\end{itemize}

\paragraph{intersect}
The intersection operation takes a \lstinline{ray} expressed in the current
surface coordinate system with point on the local plane. It propagates
the ray until it hits the first encountered part of the shape. It modifies
the ray in-place. The modified ray is still expressed in the same current
coordinate system. This operation is illustrated on
\cref{fig:abstract-rayinter}.

\begin{figure} \caption{\label{fig:abstract-rayinter} Illustration of the
abtract intersect operation for a shape on a ray.}
\includesvg[height=.2\textheight, width=.9\textwidth, keepaspectratio]
           {images/shape/abstract-rayinter.svg}
\end{figure}

\paragraph{normal}
The normal operation provides a normal vector at the ray's current
position on the shape. The normal vector is expressed in the surface
LCS. The normal vector is a unit vector. The normal vector is oriented
with a $\hat{z}$ component of opposite sign to that of the ray's vector,
\textit{ie} the normal vector is in the opposite half-plane to the incident
ray. The normal operation is illustrated on \cref{fig:abstract-normal}. Error
cases can in theory happen due to numerical issues, they should be very rare.

\begin{figure} \caption{\label{fig:abstract-normal} Illustration of the
abstract normal operation for a shape and a ray.}
\includesvg[height=.2\textheight, width=.9\textwidth, keepaspectratio]
           {images/shape/abstract-normal.svg}
\end{figure}

\subsubsection{plane}
\paragraph{Definition}
A \lstinline{plane} is the local $z=0$ plane in the current \gls{LCS}.
It is specified implicitely.

\paragraph{intersect}
The input ray is already on the local plane. We do \emph{nothing} and cannot
fail.

\paragraph{normal}
The plane normal vector is trivial.
It is $\overrightarrow{N} = (0, 0, -\textrm{sign}(n))$.
There are no error cases.

\subsubsection{sphere}

\paragraph{Definition}
The \lstinline{sphere} is specified by:

\begin{itemize}
\item $R$: radius of curvature
\end{itemize}

The sphere is defined in the \gls{LCS} by \cref{eq:sphere-def}.

\begin{equation} \label{eq:sphere-def}
x^2 + y^2 + (z - R)^2 = R^2
\end{equation}

$R$ is the signed distance $\overline{AC}$, with $C$ the sphere center
(\cref{fig:sphere-def-lcs}). Only the hemisphere closest to the local plane
is considered as part of the \lstinline{sphere} shape.

\begin{figure} \caption{\label{fig:sphere-def-lcs} Sphere definition
in the LCS. Both signs of $R$ are represented.}
\includesvg[height=.2\textheight, width=.9\textwidth, keepaspectratio]
           {images/shape/sphere.svg}
\end{figure}

\textcolor{red}{TODO:
1. What happens if we define a sphere with R = 0? and for small R?
}

\paragraph{intersect}
The intersection between the input ray and the sphere is computed as
follows. The quantities involved are illustrated on \cref{fig:sphere-inter}.

\begin{figure} \caption{\label{fig:sphere-inter} Sphere intersection with
a ray diagram. The intersection point is $I$. The diagram is drawn arbitrarily
in the case $R<0$.}
\includesvg[height=.2\textheight, width=.9\textwidth, keepaspectratio]
           {images/shape/sphere-rayinter.svg}
\end{figure}

\begin{equation} \label{eq:ray-sphere-inter1}
\begin{cases}
b &= 2 (x \cdot l + y \cdot m - n \cdot R) \\
c &= x^2 + y^2 \\
\Delta &= b^2 - 4 c
\end{cases}
\end{equation}

If $\Delta \leq 0$, then no intersection exists. This is a ray error case.
Else, we continue with \cref{eq:ray-sphere-inter-tsol}.

\begin{equation} \label{eq:ray-sphere-inter-tsol}
t_\textrm{sol} = \frac{-b + \textrm{sign}(b) \cdot \sqrt{\Delta}}{2}
\end{equation}

Let $(x_I, y_I, z_I)$ the intersection point we wish to compute.

\begin{equation}
z_I = t_\textrm{sol} \cdot n
\end{equation}

If $\abs{z_I} \geq \abs{R}$, we have an intersection in the hemisphere furthest
from the local plane. This is another ray error case. Else, the remainder of
the solution can be computed.

\begin{equation} \begin{aligned}
x_I &= x + t_\textrm{sol} \cdot l \\
y_I &= y + t_\textrm{sol} \cdot m
\end{aligned} \end{equation}

\paragraph{normal}
The sphere normal vector is computed as follows. The ray rests on the previously
computed intersection point.

\begin{equation}
\overrightarrow{N} =
\frac{\textrm{sign}(n)}{R} \cdot
\begin{bmatrix} x \\ y \\ z - R \end{bmatrix}
\end{equation}

\subsection{lpart}

\subsubsection{tfr}

\subsubsection{surf}

\section{Proofs and justification}
Some implementation details require further justification and explanations.

\subsection{Ray intersection with a sphere}
Let us detail the computation of point $I$, the intersection between
a ray and a sphere. The ray has its point $P$ on the local plane
($z_P = 0$).
$I$ is both on the ray trajectory (parametrized by $t$) and on the sphere,
hence \cref{eq:sphere-intersect-just1}.

\begin{equation} \label{eq:sphere-intersect-just1}
\begin{cases}
x^2 + y^2 + (z - R)^2 = R^2 \\
x = x_P + t \cdot l \\
y = y_P + t \cdot m \\
z = t \cdot n
\end{cases}
\end{equation}

By substitution, we obtain \cref{eq:sphere-intersect-just2}.

\begin{equation} \label{eq:sphere-intersect-just2}
\begin{split}
&{x_P}^2 + 2 x_P \cdot t \cdot l + t^2
\cdot l^2 \\
+ &{y_P}^2 + 2 y_P \cdot t \cdot m + t^2
\cdot m^2 \\
+ &t^2 \cdot n^2 - 2 R \cdot t \cdot n + R^2
= R^2 \\
\iff &{x_P}^2 + {y_P}^2 \\
+ & 2 t (x_P \cdot l + y_P \cdot m - n \cdot R) \\
+ & t^2 (l^2 + m^2 + n^2) = 0
\end{split} \end{equation}

Since $\overrightarrow{V}$ is a unit vector, $(l^2 + m^2 + n^2) = 1$.
Hence we have a quadratic equation in $t$, \cref{eq:sphere-intersect-just3}.

\begin{equation} \label{eq:sphere-intersect-just3} \begin{cases}
t^2 + b \cdot t + c = 0 \\
b = 2 (x_P \cdot l + y_P \cdot m - n \cdot R) \\
c = {x_P}^2 + {y_P}^2
\end{cases} \end{equation}

The cases in \cref{eq:sphere-intersect-just4} are distinguished.

\begin{equation} \label{eq:sphere-intersect-just4} \begin{cases}
\Delta = b^2 - 4c & \\
\Delta < 0 & \text{no intersection} \\
\Delta = 0 & \text{one intersection (the ray is tangent to the sphere)} \\
\Delta > 0 & \text{two intersections can be found}
\end{cases} \end{equation}

We discard the case where no intersection is found. We also discard
the tangency case for two reasons. First, numerically, it cannot be checked
rigorously in our framework. Second, we see no application within our scope
that would exhibit this case nominally.

The two intersections are given by \cref{eq:sphere-intersect-just5}.

\begin{equation} \label{eq:sphere-intersect-just5} \begin{cases}
t_1 = \frac{-b + \sqrt{\Delta}}{2} \\
t_2 = \frac{-b - \sqrt{\Delta}}{2}
\end{cases} \end{equation}

We want the intersection to be the one closest to the local plane
(see the justification below), hence with minimal $\abs{z}$
(\cref{eq:sphere-intersect-just6}).

\begin{equation} \label{eq:sphere-intersect-just6} \begin{cases}
t_\textrm{sol} = \underset{t}{\mathrm{argmin}} \abs{t \cdot n} 
               = \underset{t}{\mathrm{argmin}} \abs{t} \\
t = \{ t_1, t_2 \}
\end{cases} \end{equation}

Hence, $t_\textrm{sol}$ is the solution with minimal absolute value.
$\abs{-b \pm \sqrt{\Delta}}$ is minimal iff
$-b$ and $\pm \sqrt{\Delta}$ are opposite in sign. Thus our
solution is \cref{eq:sphere-intersect-just7}.

\begin{equation} \label{eq:sphere-intersect-just7}
t_\textrm{sol} = \frac{-b + \textrm{sign}(b) \cdot \sqrt{\Delta}}{2}
\end{equation}

We can start applying $t_\textrm{sol}$ to the ray trajectory to find
the intersection point.

\begin{equation}
z_I = t_\textrm{sol} \cdot n
\end{equation}

The intersection could have happened in the hemisphere which we do
not want to consider. We consider \cref{eq:sphere-intersect-just8}.

\begin{equation} \label{eq:sphere-intersect-just8}
\begin{cases}
\abs{z_I} < \abs{R} & \text{Intersection in valid hemisphere} \\
\abs{z_I} \geq \abs{R} & \text{Intersection in wrong hemisphere}
\end{cases}
\end{equation}

We then assign an error case to $\abs{z_I} \geq \abs{R}$. If the intersection
is in the right hemisphere however, we can continue by computing
the remaining point coordinates \cref{eq:sphere-intersect-just9}.

\begin{equation} \label{eq:sphere-intersect-just9}
\begin{cases}
x_I &= x + t_\textrm{sol} \cdot l \\
y_I &= y + t_\textrm{sol} \cdot m
\end{cases} \end{equation}

$\square$

\paragraph{Intersection selection rationale}
\label{sec:sphere-intersection-selection}

We explain why we select the intersection solution that is closest to
the local plane. Notably, we do not want necessarily the \emph{first}
intersection with the hemisphere encountered by the ray in its propagation.

The hemisphere is defined as either \emph{concave} or \emph{convex} by its
radius and by the orientation of incoming rays. This determines
the intended optical interaction of the ray with the surface.  For instance, a
\emph{concave} mirror applies a converging optical power to incoming rays. All
the cases are listed in \cref{tab:sphere-definition-cases}.

\begin{table} \caption{\label{tab:sphere-definition-cases} Intended sphere
shape as seen by incoming rays.}
\begin{tabular}{| c | c | c |} \hline
-       & $n < 0$ & $n > 0$ \\ \hline
$R < 0$ & convex  & concave \\ \hline
$R > 0$ & concave & convex  \\
\hline \end{tabular} \end{table}

A ray at one of its two eventual points of intersection with a sphere can be
said to either \emph{exit} or \emph{enter} the sphere, along its own
propagation direction. The shape of the sphere as seen by the oriented incoming
ray is determined by the alternative between \emph{exitting} and
\emph{entering}.  The rays incoming onto a \emph{concave} sphere must intersect
the sphere at the point they are \emph{exitting} the sphere.  Conversely, the
rays incoming onto a \emph{convex} sphere must intersect the sphere at the
point they are \emph{entering} the sphere.

Consider that: \begin{itemize}
\item Out of the two eventual intersection points, one is \emph{exitting},
the other is \emph{entering}.
\item The ray, along its propagation, must first \emph{enter} the sphere
      and then \emph{exit}.
\item For the \emph{convex} (or \emph{entering}) case, the entrance
happens closest to the local plane.
\item For the \emph{concave} (or \emph{exitting}) case, the exit
happens closest to the local plane.
\end{itemize}

Hence, the intersection point we need is always the one closest to the
local plane\footnote{Admittedly, this justification is confuse. Please
review all cases on a diagram and you can convince yourself.}.

\paragraph{b = 0 case}

\textcolor{red}{TODO:
1. What about b = 0 case? Document (already solved in notes).
2. Put a diagram for defining the quantities (placing I etc.)
3. Illustrate the error cases with diagrams.
4. Show orientation of ray has no impact on intersection.}


\section{Tests and benchmarks}
We document the rationale for tests performed on the components of the
software. We also detail a representative performance report.

\subsection{Tests}
Our rationale for testing is the following.

\begin{itemize}
\item Every function must be called at least once.
\item Every eventual error case and condition must be reached at least once.
\item The correctness is assessed on a few samples, through the means of
      either:
  \begin{itemize}
  \item Pinning the function under test to a reference function which
        is very clearly expressed with respect to the documentation.
  \item Pinning the function result to an externally computed result.
  \end{itemize}
\end{itemize}

As the development proceeds, we add cases arising from fixed bugs in order to
prevent regression.

\subsection{Performance report}
\textcolor{red}
{TODO: * Table with speeds from benchmark for each function.
       * How the benchmark is performed.
       * Why do we benchmark.}

\section{Notations}
\subsection{Multiplication and dot product}
The symbol $\cdot$ may refer to:
\begin{itemize}
\item Scalar multiplication
\item Scalar/Vector or Scalar/Matrix multiplication
\item Vector dot product
\end{itemize}

\subsection{Absolute value and 2-norm}
The unary operator notation $\abs{\cdot}$ may refer to:
\begin{itemize}
\item Absolute value of a scalar
\item 2-norm (or \emph{Euclidian} norm) of a vector
\end{itemize}



\printglossary[type=\acronymtype,style=index]
\appendix
\cleardoublepage

\apptocmd{\thebibliography}{\raggedright}{}{}
\begingroup
\setstretch{0.6}
\setlength\bibitemsep{0pt}
\printbibliography
\endgroup
\end{document}
