\section{Proofs and justification}
Some implementation details require further justification and explanations.

\subsection{Ray intersection with a sphere}
Let us detail the computation of point $I$, the intersection between
a ray and a sphere. The ray has its point $P$ on the local plane
($z_P = 0$).
$I$ is both on the ray trajectory (parametrized by $t$) and on the sphere,
hence \cref{eq:sphere-intersect-just1}.

\begin{equation} \label{eq:sphere-intersect-just1}
\begin{cases}
x^2 + y^2 + (z - R)^2 = R^2 \\
x = x_P + t \cdot l \\
y = y_P + t \cdot m \\
z = t \cdot n
\end{cases}
\end{equation}

By substitution, we obtain \cref{eq:sphere-intersect-just2}.

\begin{equation} \label{eq:sphere-intersect-just2}
\begin{split}
&{x_P}^2 + 2 x_P \cdot t \cdot l + t^2
\cdot l^2 \\
+ &{y_P}^2 + 2 y_P \cdot t \cdot m + t^2
\cdot m^2 \\
+ &t^2 \cdot n^2 - 2 R \cdot t \cdot n + R^2
= R^2 \\
\iff &{x_P}^2 + {y_P}^2 \\
+ & 2 t (x_P \cdot l + y_P \cdot m - n \cdot R) \\
+ & t^2 (l^2 + m^2 + n^2) = 0
\end{split} \end{equation}

Since $\overrightarrow{V}$ is a unit vector, $(l^2 + m^2 + n^2) = 1$.
Hence we have a quadratic equation in $t$, \cref{eq:sphere-intersect-just3}.

\begin{equation} \label{eq:sphere-intersect-just3} \begin{cases}
t^2 + b \cdot t + c = 0 \\
b = 2 (x_P \cdot l + y_P \cdot m - n \cdot R) \\
c = {x_P}^2 + {y_P}^2
\end{cases} \end{equation}

The cases in \cref{eq:sphere-intersect-just4} are distinguished.

\begin{equation} \label{eq:sphere-intersect-just4} \begin{cases}
\Delta = b^2 - 4c & \\
\Delta < 0 & \text{no intersection} \\
\Delta = 0 & \text{one intersection (the ray is tangent to the sphere)} \\
\Delta > 0 & \text{two intersections can be found}
\end{cases} \end{equation}

We discard the case where no intersection is found. We also discard
the tangency case for two reasons. First, numerically, it cannot be checked
rigorously in our framework. Second, we see no application within our scope
that would exhibit this case nominally.

The two intersections are given by \cref{eq:sphere-intersect-just5}.

\begin{equation} \label{eq:sphere-intersect-just5} \begin{cases}
t_1 = \frac{-b + \sqrt{\Delta}}{2} \\
t_2 = \frac{-b - \sqrt{\Delta}}{2}
\end{cases} \end{equation}

We want the intersection to be the one closest to the local plane
(see TODO:REF), hence with minimal $\abs{z}$ (\cref{eq:sphere-intersect-just6}).

\begin{equation} \label{eq:sphere-intersect-just6} \begin{cases}
t_\textrm{sol} = \underset{t}{\mathrm{argmin}} \abs{t \cdot n} 
               = \underset{t}{\mathrm{argmin}} \abs{t} \\
t = \{ t_1, t_2 \}
\end{cases} \end{equation}

Hence, $t_\textrm{sol}$ is the solution with minimal absolute value.
$\abs{-b \pm \sqrt{\Delta}}$ is minimal iff
$-b$ and $\pm \sqrt{\Delta}$ are opposite in sign. Thus our
solution is \cref{eq:sphere-intersect-just7}.

\begin{equation} \label{eq:sphere-intersect-just7}
t_\textrm{sol} = \frac{-b + \textrm{sign}(b) \cdot \sqrt{\Delta}}{2}
\end{equation}

\textcolor{red}{Finish.}

\paragraph{Intersection selection rationale}

\paragraph{b = 0 case}

\textcolor{red}{TODO:
1. What about b = 0 case? Document (already solved in notes).
1b. Why do we take the ray closest to z=0? This is not the
    intersection which is encountered first by the ray.
    Justify in a paragraph that we indeed want the closest to z=0
    since it is where the sphere is on the "right side".
2. Put a diagram for defining the quantities (placing I etc.)
3. Illustrate the error cases with diagrams.
4. Show orientation of ray has no impact on intersection.}

