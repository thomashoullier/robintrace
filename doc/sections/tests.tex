\section{Tests and benchmarks}
We document the rationale for tests performed on the components of the
software. We also detail a representative performance report.

\subsection{Tests}
Our rationale for testing is the following.

\begin{itemize}
\item Every function must be called at least once.
\item Every eventual error case and condition must be reached at least once.
\item The correctness is assessed on a few samples, through the means of
      either:
  \begin{itemize}
  \item Pinning the function under test to a reference function which
        is very clearly expressed with respect to the documentation.
  \item Pinning the function result to an externally computed result.
  \end{itemize}
\end{itemize}

As the development proceeds, we add cases arising from fixed bugs in order to
prevent regression.

\subsection{Performance report}
We provide benchmark results for the operations we wrote. The goal is
to indicate what order of magnitude of performance can be expected.

\subsubsection{Hypotheses and limitations}
We run each function on sizeable arrays of rays, one ray at a time.
The functions are evaluated independently. This may be unrepresentative of
a real raytracing computation flow where complex operations may be chained
on only a few rays. The operations are run on CPU on a single thread.

\subsubsection{Hardware}
We ran the benchmark on a Void Linux desktop computer, with kernel version
6.0.11\_1. We took no particular steps to configure the OS. The computer
has a AMD Athlon 3000G CPU and DDR4 RAM clocked at 2667 MT/s. We compiled
the benchmark with gcc 10.2.1 with the flags \lstinline{-march=native -O2}.

\subsubsection{Results}
The benchmark results are listed in \cref{tab:benchmark}.

\begin{table} \caption{\label{tab:benchmark} Benchmark results}
\begin{tabular} {| c | c | c |} \hline
\textbf{Operation} & \textbf{Mean timing per op (ns)} &
  \textbf{Standard deviation (ns)} \\ \hline
reflect & 2.2 & 0.3 \\ \hline
refract & 5.6 & 0.6 \\ \hline
transfer & 5.6 & 0.9 \\ \hline
standard.intersect & 5.6 & 0.6 \\ \hline
standard.normal & 5.9 & 0.9 \\ \hline
\end{tabular}\end{table}

